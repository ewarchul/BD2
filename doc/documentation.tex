\documentclass{article}
\usepackage[utf8]{inputenc}
\usepackage{amssymb}
\usepackage{cite}
\usepackage{listings}
\usepackage{algorithm2e}
\usepackage{color}
\usepackage{amsmath}
\setcounter{tocdepth}{3}
\usepackage{graphicx}
\usepackage{polski}
\usepackage{amssymb}
\newcommand*{\QEDA}{\hfill\ensuremath{\blacksquare}}%
\newcommand*{\QEDB}{\hfill\ensuremath{\square}}%<Paste>
\usepackage[a4paper]{geometry}
\usepackage{psfrag}
\usepackage{bbm}
\usepackage[T1]{fontenc}
\usepackage{color}
\usepackage{url}
\usepackage[dvipsnames]{xcolor}
\usepackage{hyperref}
\hypersetup{
    colorlinks=true,
    linkcolor=violet,
    filecolor=magenta,      
    citecolor=blue,
    urlcolor=cyan,
}
\usepackage{graphicx}
\graphicspath{{graphics/}}
\usepackage{float}
\usepackage{mathtools}
\DeclarePairedDelimiter\ceil{\lceil}{\rceil}
\DeclareMathOperator*{\argmin}{argmin}
\DeclarePairedDelimiter\floor{\lfloor}{\rfloor}

\DeclareGraphicsExtensions{.eps}
\DeclareGraphicsExtensions{.ps}
\usepackage{algorithmic}
\usepackage{psfrag}

\newcommand{\wek}[1]{
	{\bf{#1}} 
}
\newcommand{\jed}[1]{
	{$\left[#1\right]$}
}
\newcommand{\mat}[1]{
	{\bf #1} 
}
\newcommand{\todo}[1]{
	\colorbox{yellow} {{\color{red}
	\emph {TODO: #1}
}}}
\newcommand{\srednia}[1]{
	\langle #1 \rangle 
}
\usepackage{fancyhdr}
\pagestyle{fancy}
\def\lecturemark{}
\fancyhf{}
\fancyhead[L]{\lecturemark}
\fancyfoot[C]{\thepage}
\newcommand{\spr}[1]{\part{#1}\def\lecturemark{\partname\ \thepart: #1}}
\renewcommand{\partname}{}
% Let's customize \part
\usepackage{etoolbox}% for \patchcmd
\renewcommand{\thepart}{\arabic{part}}
\makeatletter
\patchcmd{\@part}{\par\nobreak}{: }{}{}
\patchcmd{\@part}{\huge}{\Large}{}{}
\makeatother
\newcommand{\argmax}{\operatornamewithlimits{argmax}} 
\title{Bazy danych 2: \\ system zarządzania i udostępniania danych o umowach o poufności}
%lista i kolejnosc na niej do ustalenia
\author{Aleksandra Dzieniszewska \\ Piotr Gawroński \\ Daniel Iwanicki \\ Eryk Warchulski\\ Prowadzący: dr inż. Michał Rudowski} 
\date{\today\\wer. 1.0}
\begin{document}
\maketitle
{\footnotesize{\tableofcontents}}
\topskip0pt
\vspace*{\fill}
\vspace*{\fill}
\section{Wprowadzenie}
 Dokument ten zawiera opis realizacji kolejnych etapów rozwiązania zadań projektowych.\\
 Sekcja (\ref{s0}) zawiera opis modelu pojęciowego, tj. opis wprowadzonych encji, ich struktury oraz relacji.
 W sekcji tej znajduje się również specyfikacja technologii, które będą użyte do realizacji dalszych etapów projektu.
 Sekcja (\ref{s1}) zawiera opis modelu logicznego, który został wygenerowany na podstawie wcześniej opisanego modelu pojęciowego. 
 Znajduje się w niej również analiza wymagań funkcjonanych oraz niefunkcjonalnych, które są pochodną przyjętych reguł biznesowych.
 W  sekcjach (\ref{s2}) oraz (\ref{s3}) znajdują się opisy fizycznego modelu danych oraz zaprojektowanych i zaimplementowanych aplikacji. \\
\section{Analiza zadania}
Celem projektu jest implementacja systemu pozwalającego na zarządzanie uprawnieniami związanymi z przetwarzaniem poufnych danych.\\
Ułatwia on administrowanie uprawnieniami dostępowymi. Pozwala on także na zarządzanie umowami podpisanymi z podmiotami zewnętrznymi - dodawaniem, usuwaniem i przeglądaniem.\\
Przechowuje on informacje o osobach mających dostęp do danych oraz osobach upoważnionych do wprowadzania zmian w tych uprawnieniach.\\
Umożliwia on nadawanie i odbieranie uprawnień do przeglądania i wykorzystania danych których dotyczy umowa. \\
Uprawnienia są przyznawane do każdej umowy lub do grupy umów niezależnie.\\
System także udostępnia informacje o posiadanych danych i osobach upoważnionych do ich przetwarzania na żądanie podmiotu którego te dane dotyczą.\\
W celu realizacji projektu zakłada się, że system powstaje na potrzeby firmy $\mathcal{FIRMA}$ i jest przeznaczony wyłącznie do użytku wewnętrznego.
\section{Etap 0. \label{s0}}
\subsection{Model pojęciowy}
\subsubsection{Diagram E-R}
\subsubsection{Opis encji}
\subsubsection{Opis związków encji}
\subsection{Specyfikacja technologii}
\section{Etap 1.\label{s1}}
\subsection{Analiza wymagań}
\subsubsection{Wymagania funkcjonalne}
\subsubsection{Wymagania niefunkcjonalne}
\subsection{Logiczny model danych}
\subsubsection{diagram logiczny}
\section{Etap 2. \label{s2}}
\todo{etap 2 do 21.05}
\subsection{Fizyczny model danych}
\subsection{Implementacja bazy danych}
\subsection{Procedura generowania danych testowych}
\subsection{Scenariusz testów}
\section{Etap3. \label{s3}}
\todo{etap 3 do 11.06}
\subsection{Implementacja aplikacji dostępowej oraz raportowej}
\subsection{Wyniki testów}
\end{document}
