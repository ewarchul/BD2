\documentclass{article}
\usepackage[utf8]{inputenc}
\usepackage{amssymb}
\usepackage{lscape}
\usepackage{cite}
\usepackage{listings}
\usepackage{algorithm2e}
\usepackage{booktabs}
\usepackage{color}
\usepackage{amsmath}
\setcounter{tocdepth}{3}
\usepackage{graphicx}
\usepackage{polski}
\usepackage{amssymb}
\newcommand*{\QEDA}{\hfill\ensuremath{\blacksquare}}%
\newcommand*{\QEDB}{\hfill\ensuremath{\square}}%<Paste>
\usepackage[a4paper]{geometry}
\usepackage{psfrag}
\usepackage{bbm}
\usepackage[T1]{fontenc}
\usepackage{color}
\usepackage{url}
\usepackage[dvipsnames]{xcolor}
\usepackage{hyperref}
\hypersetup{
    colorlinks=true,
    linkcolor=blue,
    filecolor=magenta,      
    citecolor=blue,
    urlcolor=cyan,
}
\usepackage{graphicx}
\graphicspath{{graphics/}}
\usepackage{float}
\usepackage{mathtools}
\DeclarePairedDelimiter\ceil{\lceil}{\rceil}
\DeclareMathOperator*{\argmin}{argmin}
\DeclarePairedDelimiter\floor{\lfloor}{\rfloor}

\DeclareGraphicsExtensions{.eps}
\DeclareGraphicsExtensions{.ps}
\usepackage{algorithmic}
\usepackage{psfrag}

\newcommand{\wek}[1]{
	{\bf{#1}} 
}
\newcommand{\jed}[1]{
	{$\left[#1\right]$}
}
\newcommand{\mat}[1]{
	{\bf #1} 
}
\newcommand{\todo}[1]{
	\colorbox{yellow} {{\color{red}
	\emph {TODO: #1}
}}}
\newcommand{\srednia}[1]{
	\langle #1 \rangle 
}
\usepackage{fancyhdr}
\pagestyle{fancy}
\def\lecturemark{}
\fancyhf{}
\fancyhead[L]{\lecturemark}
\fancyfoot[C]{\thepage}
\newcommand{\spr}[1]{\part{#1}\def\lecturemark{\partname\ \thepart: #1}}
\renewcommand{\partname}{}
% Let's customize \part
\usepackage{etoolbox}% for \patchcmd
\renewcommand{\thepart}{\arabic{part}}
\makeatletter
\patchcmd{\@part}{\par\nobreak}{: }{}{}
\patchcmd{\@part}{\huge}{\Large}{}{}
\makeatother
\newcommand{\argmax}{\operatornamewithlimits{argmax}} 
\title{Bazy danych 2: \\ system zarządzania i udostępniania danych o umowach o poufności}
%lista i kolejnosc na niej do ustalenia
\author{Aleksandra Dzieniszewska \\ Piotr Gawroński \\ Daniel Iwanicki \\ Eryk Warchulski\\ Prowadzący: dr inż. Michał Rudowski} 
\date{\today\\wer. 1.0}
\begin{document}
\maketitle
{\footnotesize{\tableofcontents}}
\topskip0pt
\vspace*{\fill}
\vspace*{\fill}
\section{Wprowadzenie}
 Dokument ten zawiera opis realizacji kolejnych etapów rozwiązania zadań projektowych.\\
 Sekcja (\ref{s0}) zawiera opis modelu pojęciowego, tj. opis wprowadzonych encji, ich struktury oraz relacji.
 W sekcji tej znajduje się również specyfikacja technologii, które będą użyte do realizacji dalszych etapów projektu.
 Sekcja (\ref{s1}) zawiera opis modelu logicznego, który został wygenerowany na podstawie wcześniej opisanego modelu pojęciowego. 
 Znajduje się w niej również analiza wymagań funkcjonanych oraz niefunkcjonalnych, które są pochodną przyjętych reguł biznesowych.
 W  sekcjach (\ref{s2}) oraz (\ref{s3}) znajdują się opisy fizycznego modelu danych oraz zaprojektowanych i zaimplementowanych aplikacji. \\
\section{Analiza zadania}
Celem projektu jest implementacja systemu pozwalającego na zarządzanie uprawnieniami związanymi z przetwarzaniem poufnych danych.\\
Ułatwia on administrowanie uprawnieniami dostępowymi. Pozwala on także na zarządzanie umowami podpisanymi z podmiotami zewnętrznymi - dodawaniem, usuwaniem i przeglądaniem.\\
Przechowuje on informacje o osobach mających dostęp do danych oraz osobach upoważnionych do wprowadzania zmian w tych uprawnieniach.\\
Umożliwia on nadawanie i odbieranie uprawnień do przeglądania i wykorzystania danych których dotyczy umowa. \\
Uprawnienia są przyznawane do każdej umowy lub do grupy umów niezależnie.\\
System także udostępnia informacje o posiadanych danych i osobach upoważnionych do ich przetwarzania na żądanie podmiotu którego te dane dotyczą.\\
W celu realizacji projektu zakłada się, że system powstaje na potrzeby firmy $\mathcal{F}$ i jest przeznaczony wyłącznie do użytku wewnętrznego.
\subsection{Specyfikacja technologii}
	Do zrealizowania projektu zostaną użyte następujące technologie:
		\begin{itemize}
			\item \texttt{Python 3.7}, który posłuży do napisania aplikacji dostępowej i raportowej do bazy danych
			\item \texttt{PyQt}, w którym zostanie zbudowany graficzny interfejs użytkownika
			\item \texttt{SQLite} jako silnik bazy danych.
		\end{itemize}
\newpage
\section{Etap 0. \label{s0}}
\subsection{Model pojęciowy}
\subsubsection{Diagram E-R}
Na rysunku (\ref{er1}) znajduje się powstały w ramach analizy systemowej diagram E-R. Kolorem zielonym zostały oznaczone encje słownikowe.
\begin{figure}[H]
			\centering
			\includegraphics[width=1.10\textwidth]{/Users/warbarbye/Desktop/er1.png}
			\label{er1}
			\caption{Diagram E-R.}
\end{figure}
\subsubsection{Opis encji}
%%%%
\paragraph{Nazwa encji:\\ }
Dział
\paragraph{Atrybuty\\ }
\begin{table}[H]
\begin{tabular}{|l|l|l|l|}
\hline
\textbf{Nazwa atrybutu} & \textbf{Typ}    & \textbf{Ograniczenia} & \textbf{Opis}                                      \\ \hline
Id działu               & numeryczny      & Unikalny i niepusty   & Identyfikator działu w obrębie firmy $\mathcal{F}$ \\ \hline
Nazwa działu            & łańcuch znakowy & Niepusty              & Nazwa działu                                       \\ \hline
\end{tabular}
\end{table}
\paragraph{Opis encji: \\}
Encja reprezentująca działy, z których składa się firma $\mathcal{F}$. Podmioty zewnętrzne wchodzą w interakcje 
biznesowe z działami firmy i umowy są podpisywane między działami, a podmiotem zewnętrznym.
%%%%%%%%%%%%%%%%%
\paragraph{Nazwa encji:\\ }
Pracownik
\paragraph{Atrybuty\\ }
% Please add the following required packages to your document preamble:
% \usepackage{graphicx}
\begin{table}[H]
\resizebox{\textwidth}{!}{%
\begin{tabular}{|l|l|l|l|}
\hline
\textbf{Nazwa atrybutu} & \textbf{Typ}    & \textbf{Ograniczenia} & \textbf{Opis}                                                    \\ \hline
Id pracownika           & numeryczny      & Unikalny i niepusty   & Identyfikator pracownika w obrębie działu                        \\ \hline
Imię                    & łańcuch znakowy & Niepusty              & Imię pracownika                                                  \\ \hline
Nazwisko                & łańcuch znakowy & Niepusty              & Nazwisko pracownika                                              \\ \hline
PESEL                   & łańcuch znakowy & Unikalny              & Numer PESEL pracownika                                           \\ \hline
Stanowisko              & łańcuch znakowy & brak                  & Nazwa stanowiska zajmowanego przez pracownika                    \\ \hline
Poziom dostępu          & numeryczny      & Niepusty              & Liczba ze zbioru $\{1, 2, 3, 4, 5\}$, określająca poziom dostępu \\ \hline
\end{tabular}%
}
\caption{Tabela z opisem encji.} 
\end{table}
\paragraph{Opis encji: \\}
Każdy \texttt{dział} składa się z pracowników, którzy wykonują w jego obrębie pewne zadania oraz posiadają dostęp do
umów, które są podpisywane między ich działem, a podmiotem zewnętrzym. \texttt{Pracownik} posiada stopień dostępu do danych w zależności od jego funkcji.
%====
%%%
\paragraph{Nazwa encji:\\ }
Katalog informacji
\paragraph{Atrybuty\\ }
\begin{table}[H]
\resizebox{\textwidth}{!}{%
\begin{tabular}{|l|l|l|l|}
\hline
\textbf{Nazwa atrybutu} & \textbf{Typ}    & \textbf{Ograniczenia} & \textbf{Opis}                                      \\ \hline
Id katalogu             & numeryczny      & Unikalny i niepusty   & Identyfikator katalogu                             \\ \hline
Nazwa katalogu          & łańcuch znakowy & Niepusty              & Znormalizowana nazwa katalogu.                     \\ \hline
Data utworzenia         & łańcuch znakowy & Niepusty              & Data utworzenia katalogu dla podmiotu zewnętrznego \\ \hline
\end{tabular}%
}

\caption{Tabela z opisem encji.} 
\end{table}
\paragraph{Opis encji: \\}
Encja \texttt{katalog informacji} jest kontenerem, w którym przechowywane są \texttt{katalogi umów} oraz \texttt{katalogi osób}, które są upoważnione do przetwarzania danych. Jeden \texttt{katalog informacji} jest zakładany dla \texttt{podmiotu zewnętrznego}.
%====
%%%
\paragraph{Nazwa encji:\\ }
Katalog umów
\paragraph{Atrybuty\\ }
\begin{table}[H]
\resizebox{\textwidth}{!}{%
\begin{tabular}{|l|l|l|l|}
\hline
\textbf{Nazwa atrybutu} & \textbf{Typ}    & \textbf{Ograniczenia} & \textbf{Opis}                                      \\ \hline
Id katalogu umów        & numeryczny      & Unikalny i niepusty   & Identyfikator katalogu umów                        \\ \hline
Wymagany poziom dostępu & numeryczny      & Niepusty              & Minimalny poziom dostępu pracownika do umowy       \\ \hline
Data utworzenia         & łańcuch znakowy & Niepusty              & Data utworzenia katalogu dla podmiotu zewnętrznego \\ \hline
\end{tabular}%
}

\caption{Tabela z opisem encji.} 
\end{table}
\paragraph{Opis encji: \\}
Encja \texttt{katalog umów} jest kontenerem, który gromadzi \texttt{umowy} zawierane między firmą $\mathcal{F}$, a 
\texttt{podmiotem zewnętrznym} w ramach współpracy. Dany \texttt{podmiot zewnętrzny} może mieć wiele \texttt{katalogów umów}, co wynika z faktu, że podmiot taki wchodzi w interakcje z różnymi \texttt{działami} oraz w ramach różnych projektów.
%====
%%%
\paragraph{Nazwa encji:\\ }
Katalog osób
\paragraph{Atrybuty\\ }
\begin{table}[H]
\resizebox{\textwidth}{!}{%
\begin{tabular}{|l|l|l|l|}
\hline
\textbf{Nazwa atrybutu} & \textbf{Typ}    & \textbf{Ograniczenia} & \textbf{Opis}                                      \\ \hline
Id katalogu osób        & numeryczny      & Unikalny i niepusty   & Identyfikator katalogu osób                        \\ \hline
Wymagany poziom dostępu & numeryczny      & Niepusty              & Minimalny poziom dostępu pracownika do umowy       \\ \hline
Data utworzenia         & łańcuch znakowy & Niepusty              & Data utworzenia katalogu dla podmiotu zewnętrznego \\ \hline
\end{tabular}%
}

\caption{Tabela z opisem encji.} 
\end{table}
\paragraph{Opis encji: \\}
Encja \texttt{katalog osób} jest kontenerem, który gromadzi \texttt{osoby uprawnione} -- w funkcji rodzaju uprawnienia odpowiednio -- do przeglądania/przetwarzania/modyfikacji danych, które zawierają \texttt{umowy}. 
%====
%%%
\paragraph{Nazwa encji:\\ }
Osoba uprawniona
\paragraph{Atrybuty\\ }
\begin{table}[H]
\resizebox{\textwidth}{!}{%
\begin{tabular}{|l|l|l|l|}
\hline
\textbf{Nazwa atrybutu} & \textbf{Typ}    & \textbf{Ograniczenia} & \textbf{Opis}                   \\ \hline
Numery osoby            & numeryczny      & Unikalny i niepusty   & Identyfikator osoby uprawnionej \\ \hline
Imię                    & łańcuch znakowy & Niepusty              & Imię osoby uprawnionej          \\ \hline
Nazwisko                & łańcuch znakowy & Niepusty              & Nazwisko osoby uprawnionej      \\ \hline
\end{tabular}%
}

\caption{Tabela z opisem encji.} 
\end{table}
\paragraph{Opis encji: \\}
Encja \texttt{osoba uprawniona} opisuje osobę, która może dokonać pewnych operacji na danych udostępnianych w ramach pewnej \texttt{umowy}. Może to być osoba, która jest \texttt{pracownikiem} firmy $\mathcal{F}$ lub jest reprezentantem \texttt{podmiotu zewnętrznego}.  
%====%%%%
\paragraph{Nazwa encji:\\ }
Rodzaj uprawnienia
\paragraph{Atrybuty\\ }
\begin{table}[H]
\resizebox{\textwidth}{!}{%
\begin{tabular}{|l|l|l|l|}
\hline
\textbf{Nazwa atrybutu} & \textbf{Typ}    & \textbf{Ograniczenia} & \textbf{Opis}                                                                \\ \hline
Rodzaj uprawnienia      & łańcuch znakowy & Niepusty              & Dostępne uprawnienia do danych z umów, które mogą posiadać dane osoby        \\ \hline
Opis                    & łańcuch znakowy & Niepusty              & Opis uprawnienia, tj. opis czynności, które może podjąć osoba posiadające je \\ \hline
\end{tabular}%
}

\caption{Tabela z opisem encji.} 
\end{table}

\paragraph{Opis encji: \\}
Encja słownikowa, która zawiera dostępne rodzaje uprawnień do danych oraz ich słowny opis.
%====
%%%
\paragraph{Nazwa encji:\\ }
Umowa
\paragraph{Atrybuty\\ }
\begin{table}[H]
\resizebox{\textwidth}{!}{%
\begin{tabular}{|l|l|l|l|}
\hline
\textbf{Nazwa atrybutu} & \textbf{Typ}    & \textbf{Ograniczenia} & \textbf{Opis}                \\ \hline
Id umowy                & numeryczny      & Unikalny i niepusty   & Unikalny identyfikator umowy \\ \hline
Nazwa umowy             & łańcuch znakowy & Niepusty              & Znormalizowana nazwa umowy   \\ \hline
\end{tabular}%
}

\caption{Tabela z opisem encji.} 
\end{table}
\paragraph{Opis encji: \\}
Encja reprezentująca umowę podpisywaną między firmą $\mathcal{F}$ oraz \texttt{podmiotem zewnętrznym}. \texttt{Umowa} składa sie z pozycji o różnej treści. 
%====
%%%
\paragraph{Nazwa encji:\\ }
Pozycja
\paragraph{Atrybuty\\ }

\begin{table}[H]
\resizebox{\textwidth}{!}{%
\begin{tabular}{|l|l|l|l|}
\hline
\textbf{Nazwa atrybutu} & \textbf{Typ}    & \textbf{Ograniczenia} & \textbf{Opis}             \\ \hline
Nazwa pozycji           & łańcuch znakowy      & Unikalny i niepusty   & Nazwa pozycji w umowie    \\ \hline
Treść pozycji          & łańcuch znakowy & Niepusty              & Treść pozycji danej umowy \\ \hline
\end{tabular}%
}

\caption{Tabela z opisem encji.} 
\end{table}
\paragraph{Opis encji: \\}
Składowa \texttt{umowy} o określonej nazwie oraz treści. 
%====
%%%
\paragraph{Nazwa encji:\\ }
Rodzaj umowy
\paragraph{Atrybuty\\ }

\begin{table}[H]
\resizebox{\textwidth}{!}{%
\begin{tabular}{|l|l|l|l|}
\hline
\textbf{Nazwa atrybutu} & \textbf{Typ}    & \textbf{Ograniczenia} & \textbf{Opis}                                               \\ \hline
Rodzaj umowy            & łańcuch znakowy & Niepusty              & Rodzaje umów jakie może zawierać podmiot zewnętrzny z firmą \\ \hline
\end{tabular}%
}

\caption{Tabela z opisem encji.} 
\end{table}
\paragraph{Opis encji: \\}
Encja słownikowa, która zawiera możliwe rodzaje umów, które są podpisywane między firmą $\mathcal{F}$, a \texttt{podmiotami zewnętrznymi}.
%====
%%%
\paragraph{Nazwa encji:\\ }
Podmiot zewnętrzy
\paragraph{Atrybuty\\}
\begin{table}[H]
\resizebox{\textwidth}{!}{%
\begin{tabular}{|l|l|l|l|}
\hline
\textbf{Nazwa atrybutu} & \textbf{Typ}    & \textbf{Ograniczenia} & \textbf{Opis}                                                \\ \hline
Id podmiotu             & numeryczny      & Unikalny i niepusty   & Identyfikator podmiotu prawnego, który zawiera umowy z firmą \\ \hline
Status prawny           & łańcuch znakowy & Niepusty              & Podmiot prywatny lub podmiot prawny                          \\ \hline
\end{tabular}%
}

\caption{Tabela z opisem encji.} 
\end{table}
\paragraph{Opis encji: \\}
Jest to byt, który zawarł umowę z firmą $\mathcal{F}$ i zawiera dostęp katalogu swoich umów oraz osób uprawnionych do tych umów. 
%====
%%%
\paragraph{Nazwa encji:\\ }
Osoba prawna
\paragraph{Atrybuty\\ }
\begin{table}[H]
\resizebox{\textwidth}{!}{%
\begin{tabular}{|l|l|l|l|}
\hline
\textbf{Nazwa atrybutu} & \textbf{Typ}    & \textbf{Ograniczenia} & \textbf{Opis}                 \\ \hline
NIP                     & łańcuch znakowy & Unikalny i niepusty   & Numer NIP podmiotu prawnego   \\ \hline
REGON                   & łańcuch znakowy & Unikalny              & Numer REGON podmiotu prawnego \\ \hline
Nazwa                   & łańcuch znakowy & Niepusty              & Nazwa podmiotu prawnego       \\ \hline
\end{tabular}%
}

\caption{Tabela z opisem encji.} 
\end{table}
\paragraph{Opis encji: \\}
Podtyp \texttt{podmiotu zewnętrznego}, który reprezentuje podmioty prawne niebędące osobami fizycznymi zawierające umowy z firmą $\mathcal{F}$.
%====
%%%
\paragraph{Nazwa encji:\\ }
Osoba fizyczna
\paragraph{Atrybuty\\ }
\begin{table}[H]
\resizebox{\textwidth}{!}{%
\begin{tabular}{|l|l|l|l|}
\hline
\textbf{Nazwa atrybutu} & \textbf{Typ}    & \textbf{Ograniczenia} & \textbf{Opis}            \\ \hline
Imię                    & łańcuch znakowy & Niepusty              & Imię osoby fizycznej     \\ \hline
Nazwisko                & łańcuch znakowy & Niepusty              & Nazwisko osoby fizycznej \\ \hline
PESEL                   & łańcuch znakowy & Unikalny i niepusty   & Numer PESEL osoby        \\ \hline
\end{tabular}%
}

\caption{Tabela z opisem encji.} 
\end{table}
\paragraph{Opis encji: \\}
Podtyp \texttt{podmiotu zewnętrznego}, który reprezentuje indywidualne osoby zawierające umowy z firmą $\mathcal{F}$.
%===
\subsubsection{Opis związków encji}
\begin{table}[H]
\resizebox{\textwidth}{!}{%
	\begin{tabular}{|p{2.5cm}|p{2.5cm}|p{2.5cm}|p{2.5cm}|p{2.5cm}|p{2.5cm}|p{3cm}|}
\hline
\textbf{L.p.} & Nazwa encji 1. & Nazwa encji 2. & Nazwa & Typ & Krotność & \textbf{Opis} \\ \hline
1. & Dział & Pracownik & "składa się z..." & jednostronnie obligatoryjna & $1$-$n$ & Każdy dział posiada wielu pracowników, ale istnieją pracownicy, którzy nie posiadają działu. \\ \hline
2. & Pracownik & Katalog informacji & "ma dostęp..." & jednostronnie opcjonalna & $1$-$n$ & Pracownik może posiadać dostęp do katalogu informacji, ale nie istnieją katalogi, do których nikt nie ma dostępu. \\ \hline
3. & Katalog informacji & Katalog osób & "zawiera..." & jednostronnie obligatoryjna & $1$-$n$ & Katalog informacji może składać się z wielu katalogów osób. \\ \hline
4. & Katalog informacji & Katalog umów & "zawiera..." & jednostronnie obligatoryjna & $1$-$n$ & Katalog informacji może składać się z wielu katalogów umów. \\ \hline
5. & Katalog osób & Osoba uprawniona & "składa się z..." & obustronnie obligatoryjna & $1$-$n$ & Katalog osób składa się z wielu osób uprawnionych do przeglądania danych w obrębie interakcji z podmiotem zewnętrznym. \\ \hline
6. & Katalog umów & Umowa & "gromadzi..." & obustronnie obligatoryjna & $1$-$n$ & Katalog umów gromadzi wielu umów, które są podpisywane z podmiotem zewnętrznym. \\ \hline
7. & Umowa & Pozycja & "posiada..." & obustronnie obligatoryjna & $1$-$n$ & Umowa posiada co najmniej jedną pozycję. \\ \hline
8. & Umowa & Podmiot zewnętrzny & "jest podpisana z..." & jednostronnie obligatoryjna & $1$-$n$ & Podmiot zewnętrzny może podpisać umowę z firmą. \\ \hline
9. & Rodzaj uprawnienia & Osoba uprawniona & "jest przydzialny.." & jednostronnie obligatoryjna & $1$-$n$ & Rodzaj uprawnień jest przydzielany do wielu osób uprawnionych składowanych w katalogu osób. \\ \hline
10. & Rodzaj umowy & Umowa & "jest..." & jednostronnie obligatoryjna & $1$-$n$ & Rodzaj umowy jest przydzielany do wielu umów składowanych w katalogu umów. \\ \hline
\end{tabular}%
}
\label{tab-er1}
\caption{Tabela zawierająca opisy związków encji.}
\end{table}
\section{Etap 1.\label{s1}}
\subsection{Analiza wymagań}
\subsubsection{Wymagania funkcjonalne}
\paragraph{Wprowadzanie danych\\}
W tabeli (\ref{tab1}) opisane są wymagania dotyczące możliwości udostępnianych przez system w ramach operacji tworzenia, edycji lub usunięcia danych. Zakłada się, że system udostępnia
dostęp do spisanych w tabeli funkcji na podstawie poziomu uprawnień, stanowiska oraz działu.
\begin{table}[H]
\resizebox{\textwidth}{!}{%
\begin{tabular}{|p{3.5cm}|p{3.5cm}|p{3.5cm}|p{3.5cm}|}
\hline
\textbf{L.p.} & \textbf{Nazwa}        & \textbf{Opis}                                                                          & \textbf{Priorytet} \\ \hline
\multicolumn{4}{|c|}{\textbf{Wprowadzanie danych}}                                                                                                  \\ \hline
1.            & Dodawanie pracowników & System umożliwia wprowadzanie danych nowych pracowników.                               & 1                  \\ \hline
2.            & Dodawanie umów        & System umożliwia wprowadzanie danych nowych umów                                       & 1                  \\ \hline
3.            & Dodawanie podmiotów   & System umożliwia dodawanie nowych podmiotów zewnętrznych z którymi zawierane są umowy. & 1                  \\ \hline
4.            & Usuwanie danych       & System umożliwia usuwanie danych, których przechowywanie nie jest już wymagane.        & 1                  \\ \hline
5.            & Edycja danych         & System umożliwia edycję wprowadzonych danych.                                          & 2                  \\ \hline
\end{tabular}%
}
\label{tab1}
\caption{Tabela z opisem wymogów funkcjonalnych.}
\end{table}
\paragraph{Przegląd danych \\}
Tabela (\ref{tab2}) zawiera opis oraz priorytet funkcji systemu związanych z przechowywaniem i przeglądem danych. Możliwe do przeglądania dane dotycząinterakcji biznesowych podjętych przez firmę $\mathcal{F}$ z podmiotami zewnętrznymi oraz metadane związane dotyczące liczby podmiotów zewnętrznych, liczby ich umów oraz osób uprawnionych.
\begin{table}[H]
\resizebox{\textwidth}{!}{%
\begin{tabular}{|p{3.5cm}|p{3.5cm}|p{3.5cm}|p{3.5cm}|}
\hline
\textbf{L.p.} & \textbf{Nazwa}    & \textbf{Opis}                                                                                                           & \textbf{Priorytet} \\ \hline
\multicolumn{4}{|c|}{\textbf{Przechowywanie i przeglądanie danych}}                                                                                                              \\ \hline
1.            & Przeglądanie umów & System pozwala na przeglądanie listy zawartych umów.                                                                    & 1                  \\ \hline
2.            & Sortowanie umów   & System pozwala na posortowanie umów względem ich rodzaju, danych jakich dotyczą oraz podmiotu z którym zostały zawarte. & 2                  \\ \hline
3.            & Przeglądanie osób & System pozwala na przeglądanie osób uprawnionych do przetwarzania lub przeglądania danych.                              & 1                  \\ \hline
4.            & Sortowanie osób   & System pozwala na posortowanie listy osób po uprawnieniach jakie posiadają.                                             & 2                  \\ \hline
\end{tabular}%
}
\label{tab2}
\caption{Tabela z opisem wymogów funkcjonalnych.}
\end{table}
\paragraph{System uprawnień\\}
W tabeli (\ref{tab3}) zebrane są opisy wymagań funkcjonalnych związanych z możliwościami manipulacji uprawnieniami, które posiadają użytkownicy.
Podobnie jak przy wymaganiach (\ref{tab1}) zakłada się, że funkcje systemu opisane w tej tabeli dostępne są tylko dla pewnego podzbioru 
użytkowników systemu.
\begin{table}[H]
\resizebox{\textwidth}{!}{%
\begin{tabular}{|p{3.5cm}|p{3.5cm}|p{3.5cm}|p{3.5cm}|}
\hline
\textbf{L.p.} & \textbf{Nazwa}        & \textbf{Opis}                                                              & \textbf{Priorytet} \\ \hline
\multicolumn{4}{|c|}{\textbf{Uprawnienia}}                                                                                              \\ \hline
1.            & Nadawanie uprawnień   & System pozwala na nadawanie użytkownikom różnych poziomów uprawnień        & 1                  \\ \hline
2.            & Modyfikacja uprawnień & System pozwala na odbieranie użytkownikom uprawnień                        & 1                  \\ \hline
3.            & Podgląd uprawnień     & System umożliwia wyświetlenie listy użytkowników wraz z ich uprawnieniami. & 1                  \\ \hline
\end{tabular}%
}
\label{tab3}
\caption{Tabela z opisem wymogów funkcjonalnych.}
\end{table}
\paragraph{Raportowanie\\}
Opisane wymogi funkcjonalne systemu w tabeli (\ref{tab4}) dotyczą funkcji systemu dostarczanych w ramach aplikacji raportowej.  
\begin{table}[H]
\resizebox{\textwidth}{!}{%
\begin{tabular}{|p{3.5cm}|p{3.5cm}|p{3.5cm}|p{3.5cm}|}
\hline
\textbf{L.p.} & \textbf{Nazwa}    & \textbf{Opis}                                                                                                 & \textbf{Priorytet} \\ \hline
\multicolumn{4}{|c|}{\textbf{Generowanie raportów}}                                                                                                                    \\ \hline
1.            & Generacja raportu & System pozwala generować profilowane raporty tworzone na podstawie umów podpisanych przez podmioty zewnętrzne & 1                  \\ \hline
2.            & Zapis raportu     & System umożliwia zapisanie raportu do jednego z dostępnych formatów zapisu dokumentów tekstowych              & 1                  \\ \hline
3.            & Wczytanie raportu & System umożliwia wczytanie pliku o określonym formacie raportu                                                & 3                  \\ \hline
\end{tabular}%
}
\label{tab4}
\caption{Tabela z opisem wymogów funkcjonalnych.}
\end{table}
\paragraph{Konserwacja\\}
Tabela (\ref{tab5}) zawiera opis funkcjonalności dotyczących konserwacji oraz zabezpieczenia danych gromadzonych w bazie danych.
\begin{table}[H]
\resizebox{\textwidth}{!}{%
\begin{tabular}{|p{3.5cm}|p{3.5cm}|p{3.5cm}|p{3.5cm}|}
\hline
\textbf{L.p.} & \textbf{Nazwa}  & \textbf{Opis}                                                                                                                         & \textbf{Priorytet} \\ \hline
\multicolumn{4}{|c|}{\textbf{Konserwacja danych}}                                                                                                                                            \\ \hline
1.            & Tworzenie kopii & System na żądanie użytkownika z odpowiednim poziomem uprawnień pozwala wykonać kopię zapasową zgromadzonych informacji w bazie danych & 1                  \\ \hline
\end{tabular}%
}
\label{tab5}
\caption{Tabela z opisem wymogów funkcjonalnych.}
\end{table}
\subsubsection{Wymagania niefunkcjonalne}
W poniższej tabeli (\ref{tab6}) znajduje się opis najważniejszych wymagań niefunkcjonalnych, które wyłoniły się w trakcie analizy systemowej.
\begin{table}[H]
\centering
\resizebox{\textwidth}{!}{%
\begin{tabular}{|p{4cm}|p{4cm}|p{4cm}|p{4cm}|}
\hline
\textbf{L.p.} & \textbf{Nazwa}                 & \textbf{Opis}                                                                                                                                                     & \textbf{Priorytet} \\ \hline
1.            & Kopia zapasowa                 & System powinien zachowywać prawidłowy stan przechowywanych informacji pomimo awarii baz danych.                                                                   & 1                  \\ \hline
2.            & Intuicyjny interface aplikacji & System powinien mieć intuicyjny interface, tj. taki że użytkownik po dwudniowym szkoleniu będzie mógł opanować jego funkcje.                                      & 3                  \\ \hline
3.            & Dostępność systemu             & Maksymalny czas niesprawności systemu po awarii nie powinien przekroczyć 8 godzin.                                                                                & 2                  \\ \hline
4.            & Bezpieczeństwo danych          & Dane użytkowników będą przechowywane w formie zaszyfrowanej.                                                                                                      & 1                  \\ \hline
5.            & Sprawność                      & System powinien obsługiwać co najmniej 1000 użytkowników jednocześnie.                                                                                            & 2                  \\ \hline
6.            & Testowalność                   & System posiada zbiór protokołów testowania poprawności działania komponentów. W przypadku naruszenia protokołu generowany jest stosowany komunikat informacyjny.  & 3                  \\ \hline
7.            & Wydajność odpowiedzi           & Górne ograniczenia oczekiwania użytkownika na odpowiedź systemu wynosi 3 sekundy.                                                                                 & 1                  \\ \hline
8.            & Kompatybilość                  & Dostęp do dostarczanej aplikacji możliwy jest na platformach opartych o systemu z rodziny \textbackslash{}textit\{Windows\} oraz \textbackslash{}textit\{Linux\}. & 1                  \\ \hline
\end{tabular}%
}
\label{tab6}
\caption{Tabela z opisem wymogów niefunkcjonalnych.}
\end{table}
\subsection{Logiczny model danych}
Na podstawie diagramu encji (\ref{er1}) został wygenerowany diagram logiczny, który znajduje się na rysunku (\ref{log1}). Diagram został doprowadzony do trzeciej postaci normalnej \texttt{3NF}.
\subsubsection{diagram logiczny}
\begin{figure}[H]
			\centering

			\includegraphics[width=1.10\textwidth]{/Users/warbarbye/Desktop/log1.png}
					\label{log1}
			\caption{Model logiczny.}
\end{figure}

\section{Etap 2. \label{s2}}
\todo{etap 2 do 21.05}
\subsection{Fizyczny model danych}
\subsection{Implementacja bazy danych}
\subsection{Procedura generowania danych testowych}
\subsection{Scenariusz testów}
\section{Etap 3. \label{s3}}
\todo{etap 3 do 11.06}
\subsection{Implementacja aplikacji dostępowej oraz raportowej}
\subsection{Wyniki testów}
\end{document}
